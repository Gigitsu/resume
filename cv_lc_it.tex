%%%%%%%%%%%%%%%%%%%%%%%%%%%%%%%%%%%%%%%%%
% Medium Length Professional CV
% LaTeX Template
% Version 2.0 (8/5/13)
%
% This template has been downloaded from:
% http://www.LaTeXTemplates.com
%
% Original author:
% Trey Hunner (http://www.treyhunner.com/)
%
% Important note:
% This template requires the resume.cls file to be in the same directory as the
% .tex file. The resume.cls file provides the resume style used for structuring the
% document.
%
%%%%%%%%%%%%%%%%%%%%%%%%%%%%%%%%%%%%%%%%%

%----------------------------------------------------------------------------------------
%	PACKAGES AND OTHER DOCUMENT CONFIGURATIONS
%----------------------------------------------------------------------------------------

\documentclass{resume} % Use the custom resume.cls style

\usepackage[left=0.75in,top=0.6in,right=0.75in,bottom=0.6in]{geometry} % Document margins

\name{Luigi Clemente} % Your name
\address{github.com/Gigitsu \\ Skype: luigi.clemente.g2} % Your address
\address{+39 328-540-9008 \\ luigi.clemente@gsquare.it} % Your phone number and email

\begin{document}

%----------------------------------------------------------------------------------------
%	WORK EXPERIENCE SECTION
%----------------------------------------------------------------------------------------

\begin{rSection}{Esperienze Professionali}

\begin{rSubsection}{Dottori.it}{Settembre 2019 - Presente}{Senior Software Engineer}{Milano}
\item Sviluppo e manutenzione della piattaforma web Dottori.it, uno strumento in grado di aiutare l'utente nella ricerca
del miglior medico o dentista nella propria citt\`a, facilitando il contatto tra paziente e dottore, e semplificando il
processo di prenotazione.
\item Progettazione ed implementazione di un sistema a microservizi (usando Elixir, Phoenix e Ecto).
\item Progettazione ed implementazione dei test di unit\`a (usando ExUnit).
\item Progettazione ed implementazione dei database utilizzati dai microservizi (PostgreSQL)
\item Deploy dell'applicazione su un cluster Kubernetes.
\end{rSubsection}

%------------------------------------------------
\begin{rSubsection}{ATS S.p.A.}{Ottobre 2018 - Agosto 2019}{Full Stack Developer | DevOps}{Milano}
\item Sviluppo della piattaforma BlinkS, un marketplace digitale per il trading di portafogli di crediti deteriorati.
Backend restful scritto in Scala con AkkaHttp e frontend in Angular.
\item Progettazione ed implementazione delle api rest e di componenti Angular.
\item Progettazione ed implementazione dei test di unit\`a (usando ScalaTest, ScalaMock e le macro di scala).
\item Progettazione ed implementazione della pipeline di rilascio e di build utilizzando gli strumenti di CI forniti da GitLab.
\item Configurazione dell'ambiente di test e produzione su Azure (Kubernetes, MySQL).
\item Deploy dell'applicazione su un cluster Kubernetes.
\item Messa in sicuezza dell'applicazione usando ModSecurity con CRS e regole personalizzate su ingress nginx
\end{rSubsection}

%------------------------------------------------

\begin{rSubsection}{SIA S.p.A.}{Gennaio 2018 - Settembre 2018}{Backend Developer}{Milano}
\item Sviluppo della nuova versione del message broker del sistem PagoPA\textregistered~ (chiamato ``Nodo dei Pagamenti SPC"),
un sistema reactive che usa Scala, scalaxb, Akka e Akka Cluster per gestire il flusso dei pagamenti tra gli attori coinvolti e il governo.
\item Progettazione ed implementazione dei test per il ``Nodo dei Pagamenti SPC" (usando Scalatest).
\item Progettazione ed implementazione della pipeline di rilascio delle applicazioni utilizzando Jenkins.
\item Rilascio delle applicazioni su un cluster OpenShift in ambiente di test.
\end{rSubsection}

%------------------------------------------------

\begin{rSubsection}{Exprivia S.p.A.}{Maggio 2016 - Dicembre 2017}{Lead Developer}{Molfetta}
\item Progettazione ed implementazione di un motore, usando TypeScript e Angular2,
in grado di emulare le funzionalit\`a della tecnologia Oracle Forms e perci\`o eseguire le vecchie applicazioni Oracle Forms sui browser moderni.
\item Manutenzione di uno strumento capace di convertire i file di salvataggio di Oracle Forms in file in formato JSON/TypeScript, usando Java e ANTLR.
\item Uso di git e gitflow per gestire il processo di sviluppo.
\item Progettazione ed implementazione di widget per l'home banking di Banca Mediolanum, usando Google Closure lib, D3js e C3js.
\end{rSubsection}

%------------------------------------------------

\begin{rSubsection}{ApuliaSoft S.r.l.}{Ottobre 2015 - Presente}{Senior Developer}{Bari}
\item Sviluppo di www.dolcidee.it, una comunit\`a web dove gli utenti possono condividere ricette e consigli di cucina, usando PHP e Symfony 2.
\item Progettazione ed implementazione di un bot Slack, scritto in JavaScript e pubblicato su Heroku, per aiutare a monitorare le attivit\`a e
la felicit\`a dei dipendenti attraverso delle domande fatte dal bot al termine di ogni giornata lavorativa.
\item Progettazione ed implementazione di uno strumento per ottimizzare la disposizione degli articoli sugli espositori nei supermercati, usando TypeScript e Angular2.
\end{rSubsection}

%------------------------------------------------

\begin{rSubsection}{My Special Gift S.r.l.s., Startup}{April 2013 - May 2014}{Full Stack Developer}{Bari}
\item Progettazione ed implementazione dell'applicazione web My Special Gift tramite la quale gli utenti possono creare e condividere le proprie liste regalo e farsi finanziare da amici e parenti attraverso donazioni PayPal, usando HTML, CSS e PHP con Yii.
\end{rSubsection}

%------------------------------------------------

\begin{rSubsection}{DigitalCopy S.r.l.}{August 2010 - March 2013}{Developer}{Altamura (BA)}
\item Manutenzione di un software gestionale per supermercati di piccole e medie dimensioni, usando Microsoft Access e VB6
\item Progettazione ed implementation di un software per la gestione dei registratori di cassa nei supermercati, usando C\# .NET 4 con Microsoft WPF, dapper.net e sqlite
\item Progettazione ed implementation di un software gestionale per supermercati di piccole e medie dimensioni, usando C\# .NET 4 con Microsoft WPF, nHibernate e SQLServer
\end{rSubsection}

\end{rSection}

%----------------------------------------------------------------------------------------
%	TECHNICAL STRENGTHS SECTION
%----------------------------------------------------------------------------------------

\begin{rSection}{Competenze Tecniche}

\begin{tabular}{ @{} >{\bfseries}l @{\hspace{6ex}} l }
Sistemi Operativi & Windows, MacOS, Linux \\
Principali linguaggi & Java, Scala, JavaScript, TypeScript, C\# \\
Altri linguaggi e runtime & Node.js, PHP, Python, Rust, R, SQL, HTML/XML, CSS \\
Frameworks e librerie & Apache Spark, Akka, Angular JS, D3js, C3js \\
RDBMS & MySQL, PostgreSQL, SQL Server \\
NoSQL & MongoDB, Elasticsearch, Redis \\
Altro software & Git, SBT, Eclipse, IntelliJ, Vim \\
& Visual Studio, Kibana \\
Altro & Docker, Jenkins, Microsoft Azure, Kubernetes \\
\end{tabular}

\end{rSection}

%----------------------------------------------------------------------------------------
%	EDUCATION SECTION
%----------------------------------------------------------------------------------------

\begin{rSection}{Educazione}

\begin{rSubsection}{Universit\`a degli studi di Bari ``Aldo Moro"}{Aprile 2016}{Laurea Triennale in Informatica e Tecnologie per la Produzione del Software, 110/110}{Bari}
Tesi: Applicazioni di Reti Neurali Ricorrenti al PoS-Tagging di Testi
\end{rSubsection}

\end{rSection}

%----------------------------------------------------------------------------------------
%	CERTIFICATIONS SECTION
%----------------------------------------------------------------------------------------
\begin{rSection}{Certificazioni}

\begin{rSubsection}{Coursera}{2017}{}{}
\item Functional Programming Principles in Scala (Licenza numero: 85UUFRQVJ75F)
\item Functional Program Design in Scala (Licenza numero: BYV4DHVGYYDU)
\item Parallel programming (Licenza numero: 6BX9A2UTCGAX)
\end{rSubsection}

\begin{rSubsection}{MongoDB University}{2014}{}{}
\item M101P MongoDB for Developers
\end{rSubsection}


\end{rSection}

%----------------------------------------------------------------------------------------
%	LANGUAGES SECTION
%----------------------------------------------------------------------------------------

\begin{rSection}{Languages}

\begin{tabular}{ @{} >{\bfseries}l @{\hspace{6ex}} l }
Italiano & Madrelingua \\
Inglese & Livello intermedio, scritto e parlato
\end{tabular}

\end{rSection}

%----------------------------------------------------------------------------------------
%	EXAMPLE SECTION
%----------------------------------------------------------------------------------------

%\begin{rSection}{Section Name}

%Section content\ldots

%\end{rSection}

%----------------------------------------------------------------------------------------

\end{document}
