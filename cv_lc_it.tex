%%%%%%%%%%%%%%%%%%%%%%%%%%%%%%%%%%%%%%%%%
% GG Professional CV
% LaTeX Template
% Version 0.1 (5/9/22)

%%%%%%%%%%%%%%%%%%%%%%%%%%%%%%%%%%%%%%%%%

%-----------------------------------------------------------------------------------------
%	PACKAGES AND OTHER DOCUMENT CONFIGURATIONS
%-----------------------------------------------------------------------------------------

\documentclass{resume} % Use the custom resume.cls style

\name{Luigi}{Clemente}
\position{Software Engineer}
\github{gigitsu}
\mobile{(+39) 328-5409008}
\linkedin{luigi-clemente-g2}
\email{luigi.clemente@gsquare.it}

\begin{document}

%-----------------------------------------------------------------------------------------
%	WORK EXPERIENCE SECTION
%-----------------------------------------------------------------------------------------

\cvSection{Esperienze Professionali}

%-----------------------------------------------------------------------------------------

\cvEntry
  {AgileLab}
  {Milano}
  {Data Engineer}
  {Ottobre 2019 - Presente}{
    \begin{cvEntryItems}
      \item Progettazione e sviluppo della piattaforma di DataMesh per Enel usando come
            teconologie Dremio, Spark, Springboot con linguaggi Java, Scala e Python.
    \end{cvEntryItems}
}

%-----------------------------------------------------------------------------------------

\cvEntry
  {Dottori.it}
  {Milano}
  {Back-end Software Engineer}
  {Settembre 2019 - Ottobre 2019}{
    \begin{cvEntryItems}
      \item Sviluppo e manutenzione della piattaforma web Dottori.it, uno strumento in grado
            di aiutare l'utente nella ricerca del miglior medico o dentista nella propria
            citt\`a, facilitando il contatto tra paziente e dottore, e semplificando il
            processo di prenotazione.
      \item Progettazione ed implementazione di un sistema a microservizi (usando Elixir,
            Phoenix e Ecto).
      \item Progettazione ed implementazione dei test di unit\`a (usando ExUnit).
      \item Progettazione ed implementazione dei database utilizzati dai microservizi (PostgreSQL)
      \item Deploy dell'applicazione su un cluster Kubernetes.
    \end{cvEntryItems}
}

%-----------------------------------------------------------------------------------------

\cvEntry
  {ATS S.p.A.}
  {Milano}
  {Back-end Software Engineer | DevOps (consulente esterno ApuliaSoft)}
  {Ottobre 2018 - Agosto 2019}{
    \begin{cvEntryItems}
      \item Sviluppo della piattaforma BlinkS, un marketplace digitale per il trading di
            portafogli di crediti deteriorati. Backend restful scritto in Scala con AkkaHttp
            e frontend in Angular.
      \item Progettazione ed implementazione delle api rest e di componenti Angular.
      \item Progettazione ed implementazione dei test di unit\`a (usando ScalaTest, ScalaMock
            e le macro di Scala).
      \item Progettazione ed implementazione della pipeline di rilascio e di build utilizzando
            gli strumenti di CI forniti da GitLab.
      \item Configurazione dell'ambiente di test e produzione su Azure (Kubernetes, MySQL).
      \item Deploy dell'applicazione su un cluster Kubernetes.
      \item Messa in sicuezza dell'applicazione usando ModSecurity con CRS e regole personalizzate
            su ingress nginx
    \end{cvEntryItems}
}

%-----------------------------------------------------------------------------------------

\cvEntry
  {SIA S.p.A.}
  {Milano}
  {Back-end Software Engineer (consulente esterno ApuliaSoft)}
  {Gennaio 2018 - Settembre 2018}{
    \begin{cvEntryItems}
      \item Sviluppo della nuova versione del message broker del sistem PagoPA\textregistered~
            (chiamato ``Nodo dei Pagamenti SPC"), un sistema reactive che usa Scala, scalaxb,
            Akka e Akka Cluster per gestire il flusso dei pagamenti tra gli attori coinvolti
            e il governo.
      \item Progettazione ed implementazione dei test per il ``Nodo dei Pagamenti SPC" (usando
            Scalatest).
      \item Progettazione ed implementazione della pipeline di rilascio delle applicazioni
            utilizzando Jenkins.
      \item Rilascio delle applicazioni su un cluster OpenShift in ambiente di test.
    \end{cvEntryItems}
}

%-----------------------------------------------------------------------------------------

\cvEntry
  {Exprivia S.p.A.}
  {Molfetta}
  {Software Architect (consulente esterno ApuliaSoft)}
  {Maggio 2016 - Dicembre 2017}{
    \begin{cvEntryItems}
      \item Progettazione ed implementazione di un motore, usando TypeScript e Angular2, in
            grado di emulare le funzionalit\`a della tecnologia Oracle Forms e perci\`o
            eseguire le vecchie applicazioni Oracle Forms sui browser moderni.
      \item Manutenzione di uno strumento capace di convertire i file di salvataggio di Oracle
            Forms in file in formato JSON/TypeScript, usando Java e ANTLR.
      \item Uso di git e gitflow per gestire il processo di sviluppo.
      \item Progettazione ed implementazione di widget per l'home banking di Banca Mediolanum,
            usando Google Closure lib, D3js e C3js.
    \end{cvEntryItems}
}

%-----------------------------------------------------------------------------------------

\cvEntry
  {ApuliaSoft S.r.l.}
  {Bari}
  {Senior Developer}
  {Ottobre 2015 - Agosto 2019}{
    \begin{cvEntryItems}
      \item Sviluppo di www.dolcidee.it, una comunit\`a web dove gli utenti possono condividere
            ricette e consigli di cucina, usando PHP e Symfony 2.
      \item Progettazione ed implementazione di un bot Slack, scritto in JavaScript e pubblicato
            su Heroku, per aiutare a monitorare le attivit\`a e la felicit\`a dei dipendenti
            attraverso delle domande fatte dal bot al termine di ogni giornata lavorativa.
      \item Progettazione ed implementazione di uno strumento per ottimizzare la disposizione
            degli articoli sugli espositori nei supermercati, usando TypeScript e Angular2.
    \end{cvEntryItems}
}

%-----------------------------------------------------------------------------------------

\cvEntry
  {My Special Gift S.r.l.s., Startup}
  {Bari}
  {Full Stack Software Engineer}
  {April 2013 - May 2014}{
    \begin{cvEntryItems}
      \item Progettazione ed implementazione dell'applicazione web My Special Gift tramite
            la quale gli utenti possono creare e condividere le proprie liste regalo e farsi
            finanziare da amici e parenti attraverso donazioni PayPal, usando HTML, CSS e
            PHP con Yii.
    \end{cvEntryItems}
}

%-----------------------------------------------------------------------------------------

\cvEntry
  {DigitalCopy S.r.l.}
  {Altamura (BA)}
  {Full Stack Software Engineer}
  {August 2010 - March 2013}{
    \begin{cvEntryItems}
      \item Manutenzione di un software gestionale per supermercati di piccole e medie
            dimensioni, usando Microsoft Access e VB6
      \item Progettazione ed implementation di un software per la gestione dei registratori
            di cassa nei supermercati, usando C\# .NET 4 con Microsoft WPF, dapper.net e sqlite
      \item Progettazione ed implementation di un software gestionale per supermercati di
            piccole e medie dimensioni, usando C\# .NET 4 con Microsoft WPF, nHibernate e SQLServer
    \end{cvEntryItems}
}

%-----------------------------------------------------------------------------------------

%-----------------------------------------------------------------------------------------
%	TECHNICAL STRENGTHS SECTION
%-----------------------------------------------------------------------------------------

\cvSection{Competenze}

\begin{cvSkills}

%-----------------------------------------------------------------------------------------

  \cvSkill
    {Programmazione}
    {Elixir, Scala, Java, TypeScript, C\#, PHP, Python, SQL}

%-----------------------------------------------------------------------------------------

  \cvSkill
    {Front-end}
    {HTML5, SASS, LESS, Angular}

%-----------------------------------------------------------------------------------------

  \cvSkill
    {Databases}
    {MySQL, PostgreSQL, SQL Server, MongoDB, Elasticsearch, Redis}

%-----------------------------------------------------------------------------------------

  \cvSkill
    {Back-end}
    {Phoenix, Ecto, Apache Spark, Akka}

%-----------------------------------------------------------------------------------------

  \cvSkill
    {DevOps}
    {Kibana, Docker, K8s, Microsoft Azure, AWS}

%-----------------------------------------------------------------------------------------

  \cvSkill
    {Tools}
    {Git, IntelliJ, Vim, VS Code, Tmux}

%-----------------------------------------------------------------------------------------

  \cvSkill
    {}{}

%-----------------------------------------------------------------------------------------

  \cvSkill
    {Lingue}
    {Italiano, Inglese}

%-----------------------------------------------------------------------------------------

\end{cvSkills}

%-----------------------------------------------------------------------------------------
%	EDUCATION SECTION
%-----------------------------------------------------------------------------------------

\cvSection{Educazione}

%-----------------------------------------------------------------------------------------

\cvEntry
  {Universit\`a degli studi di Bari ``Aldo Moro"}
  {Bari}
  {Laurea Triennale in Informatica e Tecnologie per la Produzione del Software, 110/110}
  {Aprile 2016}{
    \begin{cvEntryItems}
      Tesi: Applicazioni di Reti Neurali Ricorrenti al PoS-Tagging di Testi
    \end{cvEntryItems}
}

%-----------------------------------------------------------------------------------------

%-----------------------------------------------------------------------------------------
%	CERTIFICATIONS SECTION
%-----------------------------------------------------------------------------------------

\cvSection{Certificazioni}

%-----------------------------------------------------------------------------------------

\cvEntry
  {The Linux Foundation}
  {2021}{}{}{
    \begin{cvEntryItems}
      \item Certified Kubernetes Administrator (License number: LF-c55jbv7gct)
    \end{cvEntryItems}
  }

%-----------------------------------------------------------------------------------------

\cvEntry
  {Coursera}
  {2017}{}{}{
    \begin{cvEntryItems}
      \item Functional Programming Principles in Scala (License number: 85UUFRQVJ75F)
      \item Functional Program Design in Scala (License number: BYV4DHVGYYDU)
      \item Parallel programming (License number: 6BX9A2UTCGAX)
    \end{cvEntryItems}
  }

%-----------------------------------------------------------------------------------------

\cvEntry
  {MongoDB University}
  {2014}{}{}{
    \begin{cvEntryItems}
      \item M101P MongoDB for Developers
    \end{cvEntryItems}
  }

%-----------------------------------------------------------------------------------------


\end{document}
