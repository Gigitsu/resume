%%%%%%%%%%%%%%%%%%%%%%%%%%%%%%%%%%%%%%%%%
% GG Professional CV
% LaTeX Template
% Version 0.1 (5/9/22)
%
%%%%%%%%%%%%%%%%%%%%%%%%%%%%%%%%%%%%%%%%%

%-----------------------------------------------------------------------------------------
%	PACKAGES AND OTHER DOCUMENT CONFIGURATIONS
%-----------------------------------------------------------------------------------------

\documentclass{resume} % Use the custom resume.cls style

\name{Luigi}{Clemente}
\position{Software Engineer}
\github{gigitsu}
\mobile{(+39) 328-5409008}
\linkedin{luigi-clemente-g2}
\email{luigi.clemente@gsquare.it}

\begin{document}

%-----------------------------------------------------------------------------------------
%	WORK EXPERIENCE SECTION
%-----------------------------------------------------------------------------------------

\cvSection{Experience}

%-----------------------------------------------------------------------------------------
\cvEntry
  {Discoup}
  {Vicenza}
  {Software Engineer | DevOps}
  {March 2023 - Present}{
    \begin{cvEntryItems}
      \item Design and implementation of a discount and coupon management and distribution
            platform in Elixir, using frameworks such as Phoenix LiveView and Ecto,
            ensuring high reliability and performance.
      \item Development of real-time features for the user interface with LiveView,
            optimizing the user experience without relying on external frontend frameworks.
      \item Design and implementation of unit tests (using ExUnit).
      \item Design and implementation of the databases used by the microservices (PostgreSQL).
      \item Design and implementation of the infrastructure on AWS using Terraform,
            including technologies such as RDS and EKS.
      \item Automation of build, test, and deployment processes through CI/CD pipelines,
            integrated with Kubernetes (using GitHub Actions and Helm).
      \item Cross-functional collaboration with the product team to transform requirements
            into scalable technical solutions.
    \end{cvEntryItems}
}

%-----------------------------------------------------------------------------------------

\cvEntry
  {AgileLab}
  {Milan}
  {Data Engineer}
  {October 2020 - February 2023}{
    \begin{cvEntryItems}
      \item Design and development of the DataMesh platform for the Enel data warehouse,
            using technologies such as Dremio, Spark, and Spring Boot,
            with programming languages Java, Scala, and Python.
    \end{cvEntryItems}
  }

%-----------------------------------------------------------------------------------------

\cvEntry
  {Dottori.it}
  {Milan}
  {Software Engineer}
  {September 2019 - October 2020}{
    \begin{cvEntryItems}
      \item Development and maintenance of the Dottori.it web platform, a tool designed to help
            users find the best doctor or dentist in their city, facilitating contact between
            patients and doctors, and simplifying the appointment booking process.
      \item Design and implementation of a microservices architecture using Elixir, Phoenix,
            and Ecto.
      \item Design and implementation of unit tests (using ExUnit).
      \item Design and implementation of the databases used by the microservices (PostgreSQL).
      \item Deployment of the application on a Kubernetes cluster.
    \end{cvEntryItems}
}

%-----------------------------------------------------------------------------------------

\cvEntry
  {ATS S.p.A.}
  {Milan}
  {Software Engineer | DevOps (as ApuliaSoft external consultant)}
  {October 2018 - August 2019}{
    \begin{cvEntryItems}
      \item Development of BlinkS platform, a digital marketplace for non performing loans trading.
            Restful backend written in Scala with AkkaHttp and Angular for the frontend.
      \item Design and development of rest api and Angular components.
      \item Design and development of unit test (using ScalaTest, ScalaMock and Scala macro).
      \item Design and development of release and build pipelines using GitLab CI tools.
      \item Configuration of test and production environment on Azure (Kubernetes, MySQL).
      \item Application deployment on Kubernetes.
      \item Securing application using ModSecurity with CRS and custom rules on ingress nginx.
    \end{cvEntryItems}
  }

%-----------------------------------------------------------------------------------------

\cvEntry
  {SIA S.p.A.}
  {Milan}
  {Software Engineer (as ApuliaSoft external consultant)}
  {January 2018 - September 2018}{
    \begin{cvEntryItems}
      \item Development of the new message broker version of PagoPA\textregistered~system
            (named ``Nodo dei Pagamenti SPC"), a reactive system which uses Scala, scalaxb, Akka
            and Akka Cluster to handle payments flows between the involved stakeholders and the
            government.
      \item Design and development of tests for ``Nodo dei Pagamenti SPC" (using Scalatest).
      \item Design and development of application-based release pipelines with Jenkins.
      \item Release of application components on OpenShift cluster in test environment.
    \end{cvEntryItems}
  }

%-----------------------------------------------------------------------------------------

\cvEntry
  {Exprivia S.p.A.}
  {Molfetta}
  {Software Architect (as ApuliaSoft external consultant)}
  {May 2016 - December 2017}{
    \begin{cvEntryItems}
      \item Design and development of an engine, using TypeScript and Angular2, able to
            emulate Oracle Forms functionality and therefore run old Oracle Forms application
            on modern browsers.
      \item Maintenance of a tool to convert Oracle Forms save files into JSON/TypeScript
            format files, using Java and ANTLR.
      \item Use of git and gitflow to manage the development process.
      \item Design and development of widgets for the home banking of Banca Mediolanum,
            using Google Closure lib, D3js and C3js.
    \end{cvEntryItems}
  }

%-----------------------------------------------------------------------------------------

\cvEntry
  {ApuliaSoft S.r.l.}
  {Bari}
  {Software Engineer}
  {October 2015 - August 2019}{
    \begin{cvEntryItems}
      \item Development of www.dolcidee.it, a web community where people can share recipes
            and cooking advice, using PHP and Symfony 2.
      \item Design and development of a Slack bot, written in JavaScript and deployed on
            Heroku, that help monitoring activities and happiness of the employees through
            questions made by the bot at the end of every working days.
      \item Design and development of a tool to optimize the disposition of articles into
            pallet displays for supermarkets, using TypeScript and Angular2.
    \end{cvEntryItems}
  }

%-----------------------------------------------------------------------------------------

\cvEntry
  {Other}
  {Bari}
  {Software Engineer}
  {August 2010 - May 2014}{
    \begin{cvEntryItems}
      \item \textbf{My Special Gift s.r.l.s.} - Design and development of the MySpecialGift web platform for managing and sharing gift lists.
      \item \textbf{DigitalCopy s.r.l.} - Design and development of a management software for small and medium sized supermarket.
    \end{cvEntryItems}
  }

%-----------------------------------------------------------------------------------------

%-----------------------------------------------------------------------------------------
%	Skills
%-----------------------------------------------------------------------------------------

\cvSection{Skills}

\begin{cvSkills}

%-----------------------------------------------------------------------------------------

  \cvSkill
    {Programming}
    {Elixir, Scala, Java, TypeScript, C\#, PHP, Python, SQL}

%-----------------------------------------------------------------------------------------

  \cvSkill
    {Front-end}
    {HTML5, SASS, LESS, Tailwind}

%-----------------------------------------------------------------------------------------

  \cvSkill
    {Databases}
    {MySQL, PostgreSQL, SQL Server, MongoDB, Elasticsearch, Redis}

%-----------------------------------------------------------------------------------------

  \cvSkill
    {Back-end}
    {Phoenix, Ecto, Apache Spark, Akka}

%-----------------------------------------------------------------------------------------

  \cvSkill
    {DevOps}
    {Kibana, Docker, K8s, Helm, Terraform, AWS, Microsoft Azure}

%-----------------------------------------------------------------------------------------

  \cvSkill
    {Tools}
    {Git, IntelliJ, Vim, VS Code, Tmux}

%-----------------------------------------------------------------------------------------

  \cvSkill
    {}{}

%-----------------------------------------------------------------------------------------

  \cvSkill
    {Languages}
    {Italian, English}

%-----------------------------------------------------------------------------------------

\end{cvSkills}

%-----------------------------------------------------------------------------------------
%	EDUCATION SECTION
%-----------------------------------------------------------------------------------------

\cvSection{Education}

%-----------------------------------------------------------------------------------------

\cvEntry
  {University of Bari ``Aldo Moro"}
  {Bari}
  {BCs in Computer Science, 110/110}
  {April 2016}{
    \begin{cvEntryItems}
      Thesis: Applicazioni di Reti Neurali Ricorrenti al PoS-Tagging di Testi
    \end{cvEntryItems}
  }

%-----------------------------------------------------------------------------------------

%-----------------------------------------------------------------------------------------
%	CERTIFICATIONS SECTION
%-----------------------------------------------------------------------------------------

\cvSection{Certifications}

%-----------------------------------------------------------------------------------------

\cvEntry
  {The Linux Foundation}
  {2021}{}{}{
    \begin{cvEntryItems}
      \item Certified Kubernetes Administrator (License number: LF-c55jbv7gct)
    \end{cvEntryItems}
  }

%-----------------------------------------------------------------------------------------

\cvEntry
  {Coursera}
  {2017}{}{}{
    \begin{cvEntryItems}
      \item Functional Programming Principles in Scala (License number: 85UUFRQVJ75F)
      \item Functional Program Design in Scala (License number: BYV4DHVGYYDU)
      \item Parallel programming (License number: 6BX9A2UTCGAX)
    \end{cvEntryItems}
  }

%-----------------------------------------------------------------------------------------

\cvEntry
  {MongoDB University}
  {2014}{}{}{
    \begin{cvEntryItems}
      \item M101P MongoDB for Developers
    \end{cvEntryItems}
  }

%-----------------------------------------------------------------------------------------


\end{document}
